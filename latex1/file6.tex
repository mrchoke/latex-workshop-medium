\documentclass[a4paper]{article}
\usepackage[thai]{babel}
\usepackage[laksaman]{fonts-tlwg}
\usepackage[utf8x]{inputenc}

\newcommand\pali[1]{\textpali{\textbf{#1}}}
\setlength{\parindent}{0pt}

\begin{document}

\title{ตัวอย่างการใช้ภาษาบาลี}
\author{}
\date{\today}

\maketitle

\pali{หตฺเถสุ ภิกฺขเว สติ, อาทานนิกฺเขปนํ ปญฺญายติ}\\
ดูก่อนภิกษุทั้งหลาย! เมื่อมือทั้งหลายมีอยู่ การจับหรือการวางย่อมปรากฏให้เห็น

\pali{ปาเทสุ สติ, อภิกฺกมปฏิกฺกโม ปญฺญายติ}\\
เมื่อเท้าทั้งหลายมีอยู่ การก้าวไปหรือการถอยกลับย่อมปรากฏให้เห็น

\pali{ปพฺเพสุ สติ, สมฺมิญฺชนปสารณํ ปญฺญายติ}\\
เมื่อข้อศอกข้อเท้าทั้งหลายมีอยู่ การคู้เข้าหรือการเหยียดออกย่อมปรากฏให้เห็น

\pali{กุจฺฉิสฺมิํ สติ, ชิฆจฺฉาปิปาสา ปญฺญายติ}\\
เมื่อท้องไส้มีอยู่ ความหิวหรือความกระหายย่อมปรากฏให้เห็น นี้ฉันใด

\pali{เอวเมว โข ภิกฺขเว}\\
ดูก่อนภิกษุทั้งหลาย! ข้อนี้ก็ฉันนั้น

\pali{จกฺขุสมิํปิ สติ}\\
กล่าวคือ เมื่อจักษุมีอยู่

\pali{จกฺขุสมฺผสฺสปจฺจยา อุปฺปชฺชติ อชฺฌตฺตํ สุขทุกขํ}\\
เพราะจักษุสัมผัสนั้นเป็นปัจจัย ย่อมเกิดความสุขและทุกข์ขึ้นในตน

\pali{โสตสมิํ สติ}\\
เมื่อหูมีอยู่

\pali{โสตสมฺผสฺสปจฺจยา อุปฺปชฺชติ อชฺฌตฺตํ สุขทุกขํ}\\
เพราะโสตสัมผัสนั้นเป็นปัจจัย ย่อมเกิดความสุขและทุกข์ขึ้นในตน

\pali{ฆานสมิํ สติ}\\
เมื่อจมูกมีอยู่

\pali{ฆานสมฺผสฺสปจฺจยา อุปฺปชฺชติ อชฺฌตฺตํ สุขทุกขํ}\\
เพราะฆานสัมผัสนั้นเป็นปัจจัย ย่อมเกิดความสุขและทุกข์ขึ้นในตน

\pali{ชิวหาย สติ}\\
เมื่อลิ้นมีอยู่

\pali{ชิวหาสมฺผสฺสปจฺจยา อุปฺปชฺชติ อชฺฌตฺตํ สุขทุกขํ}\\
เพราะชิวหาสัมผัสนั้นเป็นปัจจัย ย่อมเกิดความสุขและทุกข์ขึ้นในตน

\pali{กายสมิํ สติ}\\
เมื่อกายมีอยู่

\pali{กายสมฺผสฺสปจฺจยา อุปฺปชฺชติ อชฺฌตฺตํ สุขทุกขํ}\\
เพราะกายสัมผัสนั้นเป็นปัจจัย ย่อมเกิดความสุขและทุกข์ขึ้นในตน

\pali{มนสมิํ สติ}\\
เมื่อใจมีอยู่

\pali{มโนสมฺผสฺสปจฺจยา อุปฺปชฺชติ อชฺฌตฺตํ สุขทุกขํ}\\
เพราะมโนสัมผัสนั้นเป็นปัจจัย ย่อมเกิดความสุขและทุกข์ขึ้นในตน ดังนี้

\pali{ทิฏฺฐา มยา ภิกฺขเว ฉ ผสฺสายตนิกา นาม นิรยา}\\
ดูก่อนภิกษุทั้งหลาย! นรก ชื่อฉผัสสายตนิกา (\emph{คือนรกอันเป็นไปในการสัมผัสทางอายตนะทั้ง ๖})
เราได้เห็นแล้ว

...

\end{document}
